%%%%%%%%%%%%%%%%%%%%%%%%%%%%%%%%%%%%%%%%%%%%%%%%%%%%%%%%%%%%%%%%%%%%%%%%%%%%%%%%%%%%%
%%%%%%%%%%%%%%%%%%%%%%%%%%%%%%%%%%%%%%%%%%%%%%%%%%%%%%%%%%%%%%%%%%%%%%%%%%%%%%%%%%%%%

\setbeamercolor{block body}{bg=blue!20}

\begin{frame}
\frametitle{HPC Borealis}
\begin{block}
	\centering
	There are two main modes Borealis supports as a compiler replacement: compilation and linking
\end{block}

\setbeamercolor{block body}{bg=red!20}
\begin{block}
	\centering
	HPC version needs to support both compilation and linking, but in distributed way
\end{block}
\end{frame}


%%%%%%%%%%%%%%%%%%%%%%%%%%%%%%%%%%%%%%%%%%%%%%%%%%%%%%%%%%%%%%%%%%%%%%%%%%%%%%%%%%%%%
%%%%%%%%%%%%%%%%%%%%%%%%%%%%%%%%%%%%%%%%%%%%%%%%%%%%%%%%%%%%%%%%%%%%%%%%%%%%%%%%%%%%%

\begin{frame}
\frametitle{Distributed compilation}
There are several ways to distributed compilation:
\begin{itemize}
	\item Compilation on the Lustre storage
	\item Distribution of intermediate build tree to the processing nodes
	\item Distribution of copies of the analyzed project
\end{itemize} 
\end{frame}

%%%%%%%%%%%%%%%%%%%%%%%%%%%%%%%%%%%%%%%%%%%%%%%%%%%%%%%%%%%%%%%%%%%%%%%%%%%%%%%%%%%%%
%%%%%%%%%%%%%%%%%%%%%%%%%%%%%%%%%%%%%%%%%%%%%%%%%%%%%%%%%%%%%%%%%%%%%%%%%%%%%%%%%%%%%

\begin{frame}
\frametitle{Distributed linking}
\begin{block}
	\centering
	We decided to distribute different SMT queries to different nodes/cores
\end{block}	
	Borealis performs analysis on an LLVM IR module, which means there are three options of how one can distribute the work:
\begin{itemize}
	\item Module level
	\item Function level
	\item Instruction level
\end{itemize}
\end{frame}

%%%%%%%%%%%%%%%%%%%%%%%%%%%%%%%%%%%%%%%%%%%%%%%%%%%%%%%%%%%%%%%%%%%%%%%%%%%%%%%%%%%%%
%%%%%%%%%%%%%%%%%%%%%%%%%%%%%%%%%%%%%%%%%%%%%%%%%%%%%%%%%%%%%%%%%%%%%%%%%%%%%%%%%%%%%

\begin{frame}
\frametitle{Distributed linking}
There are two ways how one can distribute functions between several processes:
	\begin{itemize}
		\item Dynamic distribution
		\item Static distribution
	\end{itemize}
\end{frame}

%%%%%%%%%%%%%%%%%%%%%%%%%%%%%%%%%%%%%%%%%%%%%%%%%%%%%%%%%%%%%%%%%%%%%%%%%%%%%%%%%%%%%
%%%%%%%%%%%%%%%%%%%%%%%%%%%%%%%%%%%%%%%%%%%%%%%%%%%%%%%%%%%%%%%%%%%%%%%%%%%%%%%%%%%%%

\begin{frame}
\frametitle{Dynamic distribution}
\setbeamercolor{block body}{bg=blue!20}
\begin{block}
	\centering
	Dynamic distribution is based on a single producer / multiple consumers scheme
\end{block}

\setbeamercolor{block body}{bg=red!20}
\begin{block}
	\centering
	If a process receives N functions, it also has run auxiliary LLVM passes N times
\end{block}
\end{frame}

%%%%%%%%%%%%%%%%%%%%%%%%%%%%%%%%%%%%%%%%%%%%%%%%%%%%%%%%%%%%%%%%%%%%%%%%%%%%%%%%%%%%%
%%%%%%%%%%%%%%%%%%%%%%%%%%%%%%%%%%%%%%%%%%%%%%%%%%%%%%%%%%%%%%%%%%%%%%%%%%%%%%%%%%%%%

\begin{frame}
\frametitle{Static distribution}
\setbeamercolor{block body}{bg=blue!20}
\begin{block}
	\centering
	Static distribution can be done via consistent hashing on the rank of the process in the supercomputer cluster.
\end{block}
We use the following two rank kinds:
	\begin{itemize}
		\item global rank which is same as the MPI rank
		\item local rank which uniquely identifies the process on its node
	\end{itemize}
\end{frame}

%%%%%%%%%%%%%%%%%%%%%%%%%%%%%%%%%%%%%%%%%%%%%%%%%%%%%%%%%%%%%%%%%%%%%%%%%%%%%%%%%%%%%
%%%%%%%%%%%%%%%%%%%%%%%%%%%%%%%%%%%%%%%%%%%%%%%%%%%%%%%%%%%%%%%%%%%%%%%%%%%%%%%%%%%%%

\begin{frame}
\frametitle{Improving efficiency}
To improve efficiency, we reinforce method with function complexity estimation, to balance workload between processes. We base our estimation on the following two properties:
	\begin{itemize}
		\item Function size
		\item Number of memory Store, Load and GetElementPointer instructions
	\end{itemize}
\end{frame}

%%%%%%%%%%%%%%%%%%%%%%%%%%%%%%%%%%%%%%%%%%%%%%%%%%%%%%%%%%%%%%%%%%%%%%%%%%%%%%%%%%%%%
%%%%%%%%%%%%%%%%%%%%%%%%%%%%%%%%%%%%%%%%%%%%%%%%%%%%%%%%%%%%%%%%%%%%%%%%%%%%%%%%%%%%%

\begin{frame}
\frametitle{Data synchronization}
Borealis records the analysis results for every function and does not re-analyze already processed functions. The Persistent Defect Data (PDD) is used for that and contains as entry for every encountered defect as:
	\begin{itemize}
		\item Defect location
		\item Defect type
		\item SMT result
	\end{itemize}
\setbeamercolor{block body}{bg=red!20}
\begin{block}
	\centering
	Transfer a full PDD across all Borealis instances takes a long time, so we synchronize a reduced PDD (rPDD) which is simply a list of already analyzed functions.
\end{block}
\end{frame}

%%%%%%%%%%%%%%%%%%%%%%%%%%%%%%%%%%%%%%%%%%%%%%%%%%%%%%%%%%%%%%%%%%%%%%%%%%%%%%%%%%%%%
%%%%%%%%%%%%%%%%%%%%%%%%%%%%%%%%%%%%%%%%%%%%%%%%%%%%%%%%%%%%%%%%%%%%%%%%%%%%%%%%%%%%%

\begin{frame}
\frametitle{rPDD synchronization}
To make the synchronization we utilize a two-staged approach:
	\begin{itemize}
		\item Synchronize rPDD between the processes on a single node
		\item Synchronize rPDD between the nodes
\end{itemize}
\end{frame}

%%%%%%%%%%%%%%%%%%%%%%%%%%%%%%%%%%%%%%%%%%%%%%%%%%%%%%%%%%%%%%%%%%%%%%%%%%%%%%%%%%%%%
%%%%%%%%%%%%%%%%%%%%%%%%%%%%%%%%%%%%%%%%%%%%%%%%%%%%%%%%%%%%%%%%%%%%%%%%%%%%%%%%%%%%%